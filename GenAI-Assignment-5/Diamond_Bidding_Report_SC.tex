\documentclass{article}
\usepackage{graphicx} % Required for inserting images
\usepackage[margin=0.5in]{geometry}
\usepackage{hyperref}

\title{Genetic Algorithm Approaches to Bidding in the Diamonds Card Game}
\author{Jayashre}
\date{}

\begin{document}

\maketitle

\section{Introduction}
The game "Diamonds" offers a unique blend of strategy and chance, where players vie to collect diamond cards and accumulate points. Each participant is dealt a suit of cards, excluding diamonds, and diamond cards are auctioned one by one, with players bidding using one of their own cards face down. The highest bid, corresponding to the most points, secures the diamond card. The objective is to amass the most points by the game's end. Developing robust strategies for bidding in "Diamonds" is pivotal for success. Generative AI presents an avenue for tackling this challenge by employing algorithms to optimize bidding decisions. This report delves into strategies for playing "Diamonds" using Generative AI and evaluates their efficacy.

\section{Problem Statement}
Effectively bidding in "Diamonds" entails navigating a landscape of uncertainty, competition, and strategic calculation. Players must balance the allure of acquiring valuable diamond cards with the risk of overbidding and losing points. Key challenges include assessing the value of cards, predicting opponents' bids, and adapting bidding strategies dynamically. Generative AI offers a promising solution by leveraging computational power to simulate gameplay, learn from past experiences, and refine bidding tactics. This study aims to develop AI algorithms capable of adept bidding in "Diamonds" and assess their performance against human counterparts.

\section{Teaching Generative AI the Game}
Training Generative AI to play "Diamonds" involves employing reinforcement learning techniques. The AI is immersed in simulated gameplay environments, where it learns through trial and error. Armed with knowledge of game rules, card values, and historical bidding patterns, the AI refines its strategies iteratively. Algorithms like Q-learning or deep reinforcement learning empower the AI to make informed decisions during bidding rounds. By simulating numerous game scenarios and adjusting bidding strategies based on feedback, the AI hones its bidding prowess over time.

\section{Iterating Upon Strategy}
At the outset, the AI adopts a cautious approach, weighing the value of its cards against potential bids from opponents. It employs heuristic or probabilistic methods to determine its bid, considering factors such as card point values and perceived bidding tendencies of adversaries. Through successive iterations of gameplay and reinforcement learning, the AI evolves its strategies. It learns to discern patterns in opponents' behavior, adapt to diverse game dynamics, and optimize bidding decisions for maximum point accumulation. As the AI gains experience, its bidding tactics become increasingly sophisticated and strategic.

\section{Analysis and Conclusion}
The strategies devised by the Generative AI exhibit promising performance in the realm of "Diamonds" bidding. Extensive simulations and testing demonstrate the AI's ability to rival human players, showcasing its adaptability and strategic acumen. Analysis of AI gameplay unveils insights into optimal bidding strategies, nuanced decision-making processes, and potential avenues for further enhancement. This research underscores the potential of Generative AI in refining bidding strategies in card games like "Diamonds," offering valuable contributions to the realm of artificial intelligence and game theory. As technology continues to advance, continued exploration and refinement of AI strategies hold promise for revolutionizing gameplay experiences.


\begin{itemize}
    \item \href{https://github.com/fromjyce/WE-Module3/tree/main/GenAI-Assignment-5}{Github Repo}
\end{itemize}

\end{document}
